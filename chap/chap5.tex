% vim:ts=4:sw=4
% Copyright (c) 2014 Casper Ti. Vector
% Public domain.

\chapter{总结和展望}
\section{系统测试}
为了分析和评估本论文提出的基于人脸识别的匹配算法的有效性和可用性,我们使用了该算法编写的基于婚恋数据的匹配好友的应用进行了实验测试。本实验主要真了基于不同算法的匹配系统进行了分析评估。
\subsection{数据库用户分析}
根据数据库的特性和年龄地区分布
[用户年龄分布图]
[用户地区分布图]
\subsection{实验结果}
本实验

由于在设计的每⼀一个阶段对系统的性能和可扩展性都提出了要求,最后实现的 系统具有良好的性能和可扩展性。虽然随着信息源的增多,算法耗时会变长,但接 近线性的时间复杂度使得耗时的增长处于可控的范围内,并且可以通过设置每个服 务器的服务范围使其对于每次请求的计算量维持在⼀一定的⽔水平。随着⽤用户的增多, 由于服务器可以横向扩展,只需增加服务器实例的数量便可应对更多的请求、保持一定的响应速度。

% \section{匹配算法实验}
% \subsection{根据人脸相似度的匹配}
% \subsection{根据具体人脸特征的匹配}
% \subsection{根据具体信息的匹配算法}
% \subsection{机器学习算法}


\section{展望}

\subsection{基于用户特点喜好的更进一步的匹配算法}
